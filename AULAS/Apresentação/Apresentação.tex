\documentclass{beamer}

\usetheme{Antibes}

% Loading preambulo
% PREAMBULO ====================================================================

% Packages ---------------------------------------------------------------------

% Package to Portuguese language
\usepackage[brazil]{babel}
% Package to Figures
\usepackage{graphicx}
\usepackage{tikz}
% Packages to math symbols and expressions
\usepackage{amsfonts, amssymb, amsmath}
% Package to insert code
\usepackage{listings} 
\usepackage{verbatim}
% Package to justify text
\usepackage[document]{ragged2e}
% Package to manage the bibliography
\usepackage[backend=biber, style=numeric, sorting=none]{biblatex}
% Package to facilitate quotations
\usepackage{csquotes}
% Package to use multicols
\usepackage{multicol}
% Para url
\usepackage{url}
% Fira Font
\usepackage[sfdefault, lf]{FiraSans}
% Colors
\usepackage{xcolor}
\usepackage{colortbl}
% Table
\usepackage{tabularray}
% Fill line
\usepackage{xhfill}

% Configurations ---------------------------------------------------------------

\AtBeginSection{
    \begin{frame}{\secname}
        \tableofcontents[currentsection,hideallsubsections]
    \end{frame}
}

\AtBeginSubsection{
    \begin{frame}{\subsecname}
        \tableofcontents[subsectionstyle=show/shaded/hide, subsubsectionstyle=hide]
    \end{frame}
}

\AtBeginSubsubsection{
    \begin{frame}{\subsubsecname}
        \tableofcontents[subsectionstyle=show/shaded/hide,subsubsectionstyle=show/shaded/hide/hide]
    \end{frame}
}

% Numbering slides
\setbeamertemplate{footline}[frame number]{}

% Getting rid of bottom navigation bars
\setbeamertemplate{navigation symbols}{}

% Margins
\setbeamersize{text margin left=20pt, text margin right=20pt}

% New commands -----------------------------------------------------------------

\NewTblrTableCommand \aula{\SetCell{bg=green!60,fg=white}}
\NewTblrTableCommand \prova{\SetCell{bg=red!80,fg=white}}
\NewTblrTableCommand \feriado{\SetCell{bg=blue!50,fg=white}}

\newcommand{\esta}[1]{\textbf{\underline{#1}}}

% Title
\title[AP 00]{Programação para a Web II}
% Subtitle
\subtitle{Apresentação da Disciplina}
% Author of the presentation
\author[E.J.R. Silva]{Evandro J.R. Silva}
% date of the presentation
\date{}

\begin{document}
    
\begin{frame}
    \titlepage
\end{frame}

\begin{frame}{Sumário}
    \tableofcontents
\end{frame}

%===============================================================================
% SECTION 1 ====================================================================
%===============================================================================
\section{Informações gerais}

%-------------------------------------------------------------------------------
% SUBSECTION 1.1 ---------------------------------------------------------------
%-------------------------------------------------------------------------------
\subsection{Ementa \& Objetivos}

\begin{frame}{Ementa}
    \begin{itemize}
        \justifying
        \item Comunicação entre navegador e servidor: Assíncrona e Síncrona.
        \item Segurança.
        \item Frameworks e ferramentas de desenvolvimento de aplicações web.
        \item Controle de sessão.
        \item Esquema de funcionamento de uma página web com acesso a banco de dados.
    \end{itemize}
\end{frame}

\begin{frame}{Objetivos}
    \begin{itemize}
        \justifying
        \item \textbf{Parte I} - Fundamentos de Comunicação Web e Sessões
        \item \textbf{Parte II} - Segurança e Desenvolvimento Estruturado
        \item \textbf{Parte III} - Integração com Banco de Dados
    \end{itemize}
\end{frame}

\begin{frame}{Objetivos}
    \begin{itemize}
        \justifying
        \item \textbf{Parte I} - Fundamentos de Comunicação Web e Sessões
            \begin{itemize}
                \item \textbf{Geral}
                    \begin{itemize}
                        \justifying
                        \item Compreender como ocorre a comunicação entre navegador e servidor, diferenciando os modos síncrono e assíncrono, e aplicar mecanismos de controle de sessão em aplicações web.
                    \end{itemize}
                \item \textbf{Específicos}
                    \begin{itemize}
                        \justifying
                        \item Identificar a arquitetura cliente-servidor e os protocolos básicos da Web.
                        \item Diferenciar requisições síncronas e assíncronas, entendendo suas aplicações práticas.
                        \item Utilizar Ajax e Fetch API para implementar páginas dinâmicas.
                        \item Compreender conceitos de sessão, cookies, local storage e session storage.
                        \item Implementar um sistema simples de autenticação de usuários com controle de sessão.
                    \end{itemize}
            \end{itemize}
        \item \textbf{Parte II} - Segurança e Desenvolvimento Estruturado
        \item \textbf{Parte III} - Integração com Banco de Dados
    \end{itemize}
\end{frame}

\begin{frame}{Objetivos}
    \begin{itemize}
        \justifying
        \item \textbf{Parte I} - Fundamentos de Comunicação Web e Sessões
        \item \textbf{Parte II} - Segurança e Desenvolvimento Estruturado
            \begin{itemize}
                \item \textbf{Geral}
                    \begin{itemize}
                        \justifying
                        \scriptsize
                        \item Conhecer os principais riscos de segurança em aplicações web e aplicar boas práticas de desenvolvimento, utilizando frameworks e ferramentas de apoio para estruturar aplicações seguras.
                    \end{itemize}
                \item \textbf{Específicos}
                    \begin{itemize}
                        \justifying
                        \scriptsize
                        \item Reconhecer ameaças comuns de segurança (XSS, CSRF, SQL Injection).
                        \item Aplicar HTTPS e certificados digitais para comunicação segura.
                        \item Adotar boas práticas de segurança na validação de dados em formulários.
                        \item Entender o papel dos frameworks no desenvolvimento de aplicações web.
                        \item Utilizar ferramentas de apoio (IDEs, controle de versão, bibliotecas) no processo de desenvolvimento.
                        \item Implementar aplicações seguras com frameworks modernos (ex.: Flask, Django, Express).
                    \end{itemize}
            \end{itemize}
        \item \textbf{Parte III} - Integração com Banco de Dados
    \end{itemize}
\end{frame}

\begin{frame}{Objetivos}
    \begin{itemize}
        \justifying
        \item \textbf{Parte I} - Fundamentos de Comunicação Web e Sessões
        \item \textbf{Parte II} - Segurança e Desenvolvimento Estruturado
        \item \textbf{Parte III} - Integração com Banco de Dados
            \begin{itemize}
                \item \textbf{Geral}
                    \begin{itemize}
                        \justifying
                        \scriptsize
                        \item Desenvolver aplicações web completas, capazes de se comunicar com bancos de dados relacionais, aplicando conceitos de modelagem, consultas SQL e boas práticas de integração.
                    \end{itemize}
                \item \textbf{Específicos}
                    \begin{itemize}
                        \justifying
                        \scriptsize
                        \item Entender o funcionamento de uma aplicação cliente-servidor com acesso a banco de dados.
                        \item Realizar a modelagem básica de dados com entidades e relacionamentos.
                        \item Conectar aplicações web a bancos de dados relacionais.
                        \item Executar consultas SQL (CRUD) a partir de aplicações web.
                        \item Aplicar boas práticas de segurança e eficiência no acesso a bancos de dados.
                        \item Desenvolver um projeto final integrando frontend + backend + banco de dados, aplicando os conhecimentos adquiridos durante a disciplina.
                    \end{itemize}
            \end{itemize}
    \end{itemize}
\end{frame}

%-------------------------------------------------------------------------------
% SUBSECTION 1.2 ---------------------------------------------------------------
%-------------------------------------------------------------------------------
\subsection{Bibliografia}

\begin{frame}{Bibliografia}
    \begin{itemize}
        \item \textbf{Básica - PPC}
        \begin{itemize}
            \justifying
            \scriptsize
            \item DEITEL, H. M; DEITEL, P. J; et al. Java TM: como programar. 8 ed. São Paulo: Pearson, 2010.
            \item HORTMAN, C. S.; CORNELL, G. Core Java: Volume 1. 8 ed. São Paulo: Pearson, 2010.
            \item HORTMAN, C. S.; CORNELL, G. Core Java: Volume 2. 8 ed. São Paulo: Pearson, 2010.
        \end{itemize}
        \normalsize
        \item \textbf{Bibliografia complementar - PPC}
        \begin{itemize}
            \justifying
            \scriptsize
            \item CRANE, D.; PASCARELLO, E.; DARREN, J. Ajax em Ação. 1 ed. São Paulo: Pearson, 2007.
            \item BURKE, B.; MONSON, R. Enterprise JavaBeans 3.0. 5 ed. São Paulo: Pearson, 2007.
            \item JENDROCK, E.; GOLLAPUDI, H.; SRIVATHSA. JAVA EE 6 Tutorial: The Basic Concepts. 4 ed. São Paulo: Pearson, 2011.
            \item FELKE-MORRIS, T. Basic of Web HTML5 Design \& CSS3. São Paulo: Prentice-hall, 2012.
            \item GRAHAM, S. Building WebService with Java. 2 ed. Pearson, 2005.
        \end{itemize}
    \end{itemize}
\end{frame}

\begin{frame}{Bibliografia}
    \begin{itemize}
        \item \textbf{Básica - Livros/Fontes mais recentes}
        \begin{itemize}
            \justifying
            \scriptsize
            \item HAVERBEKE, Marijn. Eloquent JavaScript. 4 ed. No starch press. Disponível em \url{https://eloquentjavascript.net/}. Acesso em 21 ago. 2025.
            \item GRINBERG, Miguel. Flask Web Development: Developing Web Applications with Python. 2 ed. O'Reilly Media, 2018.
            \item Mozilla Developer Network (MDN Web Docs). Disponível em \url{https://developer.mozilla.org/pt-BR/}. Acesso em 21 ago. 2025.
        \end{itemize}
        \normalsize
        \item \textbf{Bibliografia complementar - Livros/Fontes mais recentes}
        \begin{itemize}
            \justifying
            \scriptsize
            \item PILGRIM, Mark. Dive into HTML5 with illustrations from the public domain. Disponível em \url{https://mislav.github.io/diveintohtml5/} (HTML, contéudo em Inglês), \url{https://www.jesusda.com/docs/ebooks/ebook_manual_en_dive-into-html5.pdf} (PDF), ou \url{https://github.com/zenorocha/diveintohtml5} (conteúdo traduzido). Acesso a todas as páginas citadas em 20 ago. 2025.
            \item OWASP Application Security Verification Standard (ASVS). Disponível em \url{https://owasp.org/www-project-application-security-verification-standard/}. Acesso em 21 ago. 2025.
            \item SQL Tutorial. Disponível em \url{https://www.sqltutorial.org/}. Acesso em 21 ago. 2025.
        \end{itemize}
    \end{itemize}
\end{frame}

%-------------------------------------------------------------------------------
% SUBSECTION 1.3 ---------------------------------------------------------------
%-------------------------------------------------------------------------------
\subsection{Conteúdo Programático}

\begin{frame}{Conteúdo Programático}
    \begin{itemize}
        \item \textbf{Parte I - Fundamentos de Comunicação Web e Sessões}
            \begin{enumerate}
                \justifying
                \item \textbf{Arquitetura da Web}: Navegador, servidor, protocolos HTTP/HTTPS.
                \item \textbf{Comunicação síncrona e assíncrona}: conceitos, exemplos práticos com requisições HTTP.
                \item \textbf{Ajax e Fetch API}: diferenças, aplicações em páginas dinâmicas.
                \item\textbf{ Controle de sessão – fundamentos}: cookies, session storage, local storage.
                \item \textbf{Autenticação e gerenciamento de usuários}: login, logout, persistência de sessão.
            \end{enumerate}
    \end{itemize}
\end{frame}

\begin{frame}{Conteúdo Programático}
    \begin{itemize}
        \justifying
        \item \textbf{Parte II - Segurança e Desenvolvimento Estruturado}
            \begin{enumerate}
                \justifying
                \item \textbf{Conceitos de segurança na Web}: ameaças comuns (XSS, CSRF, SQL Injection).
                \item \textbf{HTTPS e certificados digitais}: funcionamento e importância.
                \item \textbf{Boas práticas de segurança em formulários}: validação no cliente e no servidor.
                \item \textbf{Introdução a frameworks de desenvolvimento}: o papel dos frameworks, vantagens.
                \item \textbf{Ferramentas de apoio ao desenvolvimento}: IDEs, sistemas de versionamento, bibliotecas auxiliares.
            \end{enumerate}
    \end{itemize}
\end{frame}

\begin{frame}{Conteúdo Programático}
    \begin{itemize}
        \justifying
        \item \textbf{Parte III - Integração com Banco de Dados}
            \begin{enumerate}
                \justifying
                \item \textbf{Funcionamento de uma aplicação com banco de dados}: modelo cliente-servidor.
                \item \textbf{Modelagem básica de dados}: entidades, atributos e relacionamentos.
                \item \textbf{Conexão servidor–banco de dados}: drivers, bibliotecas de acesso.
                \item \textbf{Consultas SQL em aplicações web}: SELECT, INSERT, UPDATE, DELETE.
                \item \textbf{Boas práticas na integração banco de dados – servidor web}: segurança e desempenho.
            \end{enumerate}
    \end{itemize}
\end{frame}

%-------------------------------------------------------------------------------
% SUBSECTION 1.3 ---------------------------------------------------------------
%-------------------------------------------------------------------------------
\subsection{Avaliação}

\begin{frame}{Avaliação}
    \begin{itemize}
        \justifying
        \item Ao \textbf{fim de cada unidade}, será realizada uma \textbf{avaliação parcial} dos conteúdos ministrados durante o curso da unidade, \alert{\textbf{totalizando em 03 (três) avaliações}}.
        \item A \textbf{nota de cada avaliação} poderá ser \textbf{composta por um ou mais instrumentos de avaliação}, de acordo com um dos seguintes casos:
        \begin{enumerate}
            \justifying
            \item Uma prova escrita;
            \item Um ou mais trabalhos (individuais ou em grupo);
            \item Um ou mais trabalhos, mais uma prova escrita.
        \end{enumerate}
    \end{itemize}
\end{frame}

\begin{frame}{Avaliação}
    \begin{itemize}
        \justifying
        \item Nos casos em que a \textbf{avaliação} for \textbf{composta por mais de um instrumento}, será realizado o \textbf{somatório} ou a \textbf{média ponderada} das \textbf{notas obtidas em cada instrumento} para compor a \textbf{nota final} de uma \textbf{avaliação parcial}.
        \item Os instrumentos a serem utilizados em cada avaliação serão definidos e informados no decorrer do curso.
    \end{itemize}
\end{frame}

\begin{frame}{Avaliação}
    \begin{itemize}
        \justifying
        \item As \textbf{notas} obedecem a uma escala de \textbf{0,0 (zero)} a \textbf{10,0 (dez)}, contando até a primeira ordem decimal com possíveis arredondamentos.
        \item Considerar-se-á \textbf{aprovado} na disciplina o aluno que obtiver \textbf{assiduidade igual ou superior a 75\%} e a \textbf{média aritmética} nas \underline{avaliações parciais (média parcial)} \textbf{igual ou superior a 7,0 (sete)}
        \begin{itemize}
            \justifying
            \item OU que se submeta a \alert<2>{exame final} e obtenha média aritmética entre a média parcial e exame final (média final) igual ou superior a 6,0 (seis).
            \begin{itemize}
                \justifying
                \item<2> Terá direito de realizar exame final o aluno que satisfaça os requisitos de assiduidade e que obtenha média parcial maior ou igual a 4,0 (quatro) e menor que 7,0 (sete).
            \end{itemize}
        \end{itemize}
    \end{itemize}
\end{frame}

%-------------------------------------------------------------------------------
% SUBSECTION 1.4 ---------------------------------------------------------------
%-------------------------------------------------------------------------------
\subsection{Calendário}

\begin{frame}{Calendário}
    \centering
    \begin{tblr}{c c c}
        \aula AULA & \feriado FERIADO & \prova AVALIAÇÃO
    \end{tblr}
    
    \begin{columns}
        \begin{column}{0.3\textwidth}
            \begin{table}
                \centering
                \textbf{SETEMBRO}\\ \vspace{0.15cm}
                \begin{tblr}{Q[c,m] Q[c,m] Q[c,m] Q[c,m] Q[c,m]}
                    \hline
                    \textbf{S} & \textbf{T} & \textbf{Q} & \textbf{Q} & \textbf{S} \\
                    \hline
                    01 & 02 & 03 & \aula\esta{04} & 05\\
                    08 & 09 & \aula10 & \aula11 & 12\\
                    15 & 16 & \aula17 & \aula18 & 19\\
                    22 & 23 & \aula24 & \prova25 & 26\\
                    29 & 30   &    &    &   \\
                    \hline
                \end{tblr}
            \end{table}
        \end{column}
        
        \begin{column}{0.7\textwidth}
            \begin{itemize}
                \justifying
                \item Apresentação da disciplina
            \end{itemize}
        \end{column}
    \end{columns}
\end{frame}

\begin{frame}{Calendário}
    \centering
    \begin{tblr}{c c c}
        \aula AULA & \feriado FERIADO & \prova AVALIAÇÃO
    \end{tblr}
    
    \begin{columns}
        \begin{column}{0.3\textwidth}
            \begin{table}
                \centering
                \textbf{SETEMBRO}\\ \vspace{0.15cm}
                \begin{tblr}{Q[c,m] Q[c,m] Q[c,m] Q[c,m] Q[c,m]}
                    \hline
                    \textbf{S} & \textbf{T} & \textbf{Q} & \textbf{Q} & \textbf{S} \\
                    \hline
                    01 & 02 & 03 & \aula04 & 05\\
                    08 & 09 & \aula\esta{10} & \aula11 & 12\\
                    15 & 16 & \aula17 & \aula18 & 19\\
                    22 & 23 & \aula24 & \prova25 & 26\\
                    29 & 30   &    &    &   \\
                    \hline
                \end{tblr}
            \end{table}
        \end{column}
        
        \begin{column}{0.7\textwidth}
            \begin{itemize}
                \justifying
                \item \textbf{Arquitetura da Web}: Navegador, servidor, protocolos HTTP/HTTPS.
                \item \textbf{Comunicação síncrona e assíncrona}: conceitos, exemplos práticos com requisições HTTP.
            \end{itemize}
        \end{column}
    \end{columns}
\end{frame}

\begin{frame}{Calendário}
    \centering
    \begin{tblr}{c c c}
        \aula AULA & \feriado FERIADO & \prova AVALIAÇÃO
    \end{tblr}
    
    \begin{columns}
        \begin{column}{0.3\textwidth}
            \begin{table}
                \centering
                \textbf{SETEMBRO}\\ \vspace{0.15cm}
                \begin{tblr}{Q[c,m] Q[c,m] Q[c,m] Q[c,m] Q[c,m]}
                    \hline
                    \textbf{S} & \textbf{T} & \textbf{Q} & \textbf{Q} & \textbf{S} \\
                    \hline
                    01 & 02 & 03 & \aula04 & 05\\
                    08 & 09 & \aula10 & \aula\esta{11} & 12\\
                    15 & 16 & \aula17 & \aula18 & 19\\
                    22 & 23 & \aula24 & \prova25 & 26\\
                    29 & 30   &    &    &   \\
                    \hline
                \end{tblr}
            \end{table}
        \end{column}
        
        \begin{column}{0.7\textwidth}
            \begin{itemize}
                \justifying
                \item \textbf{Ajax e Fetch API}: diferenças, aplicações em páginas dinâmicas.
            \end{itemize}
        \end{column}
    \end{columns}
\end{frame}

\begin{frame}{Calendário}
    \centering
    \begin{tblr}{c c c}
        \aula AULA & \feriado FERIADO & \prova AVALIAÇÃO
    \end{tblr}
    
    \begin{columns}
        \begin{column}{0.3\textwidth}
            \begin{table}
                \centering
                \textbf{SETEMBRO}\\ \vspace{0.15cm}
                \begin{tblr}{Q[c,m] Q[c,m] Q[c,m] Q[c,m] Q[c,m]}
                    \hline
                    \textbf{S} & \textbf{T} & \textbf{Q} & \textbf{Q} & \textbf{S} \\
                    \hline
                    01 & 02 & 03 & \aula04 & 05\\
                    08 & 09 & \aula10 & \aula11 & 12\\
                    15 & 16 & \aula\esta{17} & \aula18 & 19\\
                    22 & 23 & \aula24 & \prova25 & 26\\
                    29 & 30   &    &    &   \\
                    \hline
                \end{tblr}
            \end{table}
        \end{column}
        
        \begin{column}{0.7\textwidth}
            \begin{itemize}
                \justifying
                \item \textbf{Controle de sessão – fundamentos}: cookies, session storage, local storage.
            \end{itemize}
        \end{column}
    \end{columns}
\end{frame}

\begin{frame}{Calendário}
    \centering
    \begin{tblr}{c c c}
        \aula AULA & \feriado FERIADO & \prova AVALIAÇÃO
    \end{tblr}
    
    \begin{columns}
        \begin{column}{0.3\textwidth}
            \begin{table}
                \centering
                \textbf{SETEMBRO}\\ \vspace{0.15cm}
                \begin{tblr}{Q[c,m] Q[c,m] Q[c,m] Q[c,m] Q[c,m]}
                    \hline
                    \textbf{S} & \textbf{T} & \textbf{Q} & \textbf{Q} & \textbf{S} \\
                    \hline
                    01 & 02 & 03 & \aula04 & 05\\
                    08 & 09 & \aula10 & \aula11 & 12\\
                    15 & 16 & \aula17 & \aula\esta{18} & 19\\
                    22 & 23 & \aula24 & \prova25 & 26\\
                    29 & 30   &    &    &   \\
                    \hline
                \end{tblr}
            \end{table}
        \end{column}
        
        \begin{column}{0.7\textwidth}
            \begin{itemize}
                \justifying
                \item \textbf{Autenticação e gerenciamento de usuários}: login, logout, persistência de sessão.
            \end{itemize}
        \end{column}
    \end{columns}
\end{frame}

\begin{frame}{Calendário}
    \centering
    \begin{tblr}{c c c}
        \aula AULA & \feriado FERIADO & \prova AVALIAÇÃO
    \end{tblr}
    
    \begin{columns}
        \begin{column}{0.3\textwidth}
            \begin{table}
                \centering
                \textbf{SETEMBRO}\\ \vspace{0.15cm}
                \begin{tblr}{Q[c,m] Q[c,m] Q[c,m] Q[c,m] Q[c,m]}
                    \hline
                    \textbf{S} & \textbf{T} & \textbf{Q} & \textbf{Q} & \textbf{S} \\
                    \hline
                    01 & 02 & 03 & \aula04 & 05\\
                    08 & 09 & \aula10 & \aula11 & 12\\
                    15 & 16 & \aula17 & \aula18 & 19\\
                    22 & 23 & \aula\esta{24} & \prova25 & 26\\
                    29 & 30   &    &    &   \\
                    \hline
                \end{tblr}
            \end{table}
        \end{column}
        
        \begin{column}{0.7\textwidth}
            \begin{itemize}
                \justifying
                \item Prática.
            \end{itemize}
        \end{column}
    \end{columns}
\end{frame}

\begin{frame}{Calendário}
    \centering
    \begin{tblr}{c c c}
        \aula AULA & \feriado FERIADO & \prova AVALIAÇÃO
    \end{tblr}
    
    \begin{columns}
        \begin{column}{0.3\textwidth}
            \begin{table}
                \centering
                \textbf{SETEMBRO}\\ \vspace{0.15cm}
                \begin{tblr}{Q[c,m] Q[c,m] Q[c,m] Q[c,m] Q[c,m]}
                    \hline
                    \textbf{S} & \textbf{T} & \textbf{Q} & \textbf{Q} & \textbf{S} \\
                    \hline
                    01 & 02 & 03 & \aula04 & 05\\
                    08 & 09 & \aula10 & \aula11 & 12\\
                    15 & 16 & \aula17 & \aula18 & 19\\
                    22 & 23 & \aula24 & \prova\esta{25} & 26\\
                    29 & 30   &    &    &   \\
                    \hline
                \end{tblr}
            \end{table}
        \end{column}
        
        \begin{column}{0.7\textwidth}
            \Large\centering Primeira Avaliação
        \end{column}
    \end{columns}
\end{frame}

\begin{frame}{Calendário}
    \centering
    \begin{tblr}{c c c}
        \aula AULA & \feriado FERIADO & \prova AVALIAÇÃO
    \end{tblr}
    
    \begin{columns}
        \begin{column}{0.3\textwidth}
            \begin{table}
                \centering
                \textbf{OUTUBRO}\\ \vspace{0.15cm}
                \begin{tblr}{Q[c,m] Q[c,m] Q[c,m] Q[c,m] Q[c,m]}
                    \hline
                    \textbf{S} & \textbf{T} & \textbf{Q} & \textbf{Q} & \textbf{S} \\
                    \hline
                    &  & \aula\esta{01} & \aula02 & 03\\
                    06 & 07 & \aula08 & \aula09 & 10\\
                    13 & 14 & \feriado15 & \aula16 & 17\\
                    20 & 21 & \aula22 & \aula23 & 24\\
                    27 & 28 & \aula29 & \prova30 & 31\\
                    \hline
                \end{tblr}
            \end{table}
        \end{column}
        
        \begin{column}{0.7\textwidth}
            \begin{itemize}
                \justifying
                \item \textbf{Conceitos de segurança na Web}: ameaças comuns (XSS, CSRF, SQL Injection).
            \end{itemize}
        \end{column}
    \end{columns}
\end{frame}

\begin{frame}{Calendário}
    \centering
    \begin{tblr}{c c c}
        \aula AULA & \feriado FERIADO & \prova AVALIAÇÃO
    \end{tblr}
    
    \begin{columns}
        \begin{column}{0.3\textwidth}
            \begin{table}
                \centering
                \textbf{OUTUBRO}\\ \vspace{0.15cm}
                \begin{tblr}{Q[c,m] Q[c,m] Q[c,m] Q[c,m] Q[c,m]}
                    \hline
                    \textbf{S} & \textbf{T} & \textbf{Q} & \textbf{Q} & \textbf{S} \\
                    \hline
                    &  & \aula01 & \aula\esta{02} & 03\\
                    06 & 07 & \aula08 & \aula09 & 10\\
                    13 & 14 & \feriado15 & \aula16 & 17\\
                    20 & 21 & \aula22 & \aula23 & 24\\
                    27 & 28 & \aula29 & \prova30 & 31\\
                    \hline
                \end{tblr}
            \end{table}
        \end{column}
        
        \begin{column}{0.7\textwidth}
            \begin{itemize}
                \justifying
                \item \textbf{HTTPS e certificados digitais}: funcionamento e importância.
            \end{itemize}
        \end{column}
    \end{columns}
\end{frame}

\begin{frame}{Calendário}
    \centering
    \begin{tblr}{c c c}
        \aula AULA & \feriado FERIADO & \prova AVALIAÇÃO
    \end{tblr}
    
    \begin{columns}
        \begin{column}{0.3\textwidth}
            \begin{table}
                \centering
                \textbf{OUTUBRO}\\ \vspace{0.15cm}
                \begin{tblr}{Q[c,m] Q[c,m] Q[c,m] Q[c,m] Q[c,m]}
                    \hline
                    \textbf{S} & \textbf{T} & \textbf{Q} & \textbf{Q} & \textbf{S} \\
                    \hline
                    &  & \aula01 & \aula02 & 03\\
                    06 & 07 & \aula\esta{08} & \aula09 & 10\\
                    13 & 14 & \feriado15 & \aula16 & 17\\
                    20 & 21 & \aula22 & \aula23 & 24\\
                    27 & 28 & \aula29 & \prova30 & 31\\
                    \hline
                \end{tblr}
            \end{table}
        \end{column}
        
        \begin{column}{0.7\textwidth}
            \begin{itemize}
                \justifying
                \item Prática
            \end{itemize}
        \end{column}
    \end{columns}
\end{frame}

\begin{frame}{Calendário}
    \centering
    \begin{tblr}{c c c}
        \aula AULA & \feriado FERIADO & \prova AVALIAÇÃO
    \end{tblr}
    
    \begin{columns}
        \begin{column}{0.3\textwidth}
            \begin{table}
                \centering
                \textbf{OUTUBRO}\\ \vspace{0.15cm}
                \begin{tblr}{Q[c,m] Q[c,m] Q[c,m] Q[c,m] Q[c,m]}
                    \hline
                    \textbf{S} & \textbf{T} & \textbf{Q} & \textbf{Q} & \textbf{S} \\
                    \hline
                    &  & \aula01 & \aula02 & 03\\
                    06 & 07 & \aula08 & \aula\esta{09} & 10\\
                    13 & 14 & \feriado15 & \aula16 & 17\\
                    20 & 21 & \aula22 & \aula23 & 24\\
                    27 & 28 & \aula29 & \prova30 & 31\\
                    \hline
                \end{tblr}
            \end{table}
        \end{column}
        
        \begin{column}{0.7\textwidth}
            \begin{itemize}
                \justifying
                \item \textbf{Boas práticas de segurança em formulários}: validação no cliente e no servidor.
            \end{itemize}
        \end{column}
    \end{columns}
\end{frame}

\begin{frame}{Calendário}
    \centering
    \begin{tblr}{c c c}
        \aula AULA & \feriado FERIADO & \prova AVALIAÇÃO
    \end{tblr}
    
    \begin{columns}
        \begin{column}{0.3\textwidth}
            \begin{table}
                \centering
                \textbf{OUTUBRO}\\ \vspace{0.15cm}
                \begin{tblr}{Q[c,m] Q[c,m] Q[c,m] Q[c,m] Q[c,m]}
                    \hline
                    \textbf{S} & \textbf{T} & \textbf{Q} & \textbf{Q} & \textbf{S} \\
                    \hline
                    &  & \aula01 & \aula02 & 03\\
                    06 & 07 & \aula08 & \aula09 & 10\\
                    13 & 14 & \feriado15 & \aula\esta{16} & 17\\
                    20 & 21 & \aula22 & \aula23 & 24\\
                    27 & 28 & \aula29 & \prova30 & 31\\
                    \hline
                \end{tblr}
            \end{table}
        \end{column}
        
        \begin{column}{0.7\textwidth}
            \begin{itemize}
                \justifying
                \item \textbf{Introdução a frameworks de desenvolvimento}: o papel dos frameworks, vantagens.
            \end{itemize}
        \end{column}
    \end{columns}
\end{frame}

\begin{frame}{Calendário}
    \centering
    \begin{tblr}{c c c}
        \aula AULA & \feriado FERIADO & \prova AVALIAÇÃO
    \end{tblr}
    
    \begin{columns}
        \begin{column}{0.3\textwidth}
            \begin{table}
                \centering
                \textbf{OUTUBRO}\\ \vspace{0.15cm}
                \begin{tblr}{Q[c,m] Q[c,m] Q[c,m] Q[c,m] Q[c,m]}
                    \hline
                    \textbf{S} & \textbf{T} & \textbf{Q} & \textbf{Q} & \textbf{S} \\
                    \hline
                    &  & \aula01 & \aula02 & 03\\
                    06 & 07 & \aula08 & \aula09 & 10\\
                    13 & 14 & \feriado15 & \aula16 & 17\\
                    20 & 21 & \aula\esta{22} & \aula23 & 24\\
                    27 & 28 & \aula29 & \prova30 & 31\\
                    \hline
                \end{tblr}
            \end{table}
        \end{column}
        
        \begin{column}{0.7\textwidth}
            \begin{itemize}
                \justifying
                \item Prática.
            \end{itemize}
        \end{column}
    \end{columns}
\end{frame}

\begin{frame}{Calendário}
    \centering
    \begin{tblr}{c c c}
        \aula AULA & \feriado FERIADO & \prova AVALIAÇÃO
    \end{tblr}
    
    \begin{columns}
        \begin{column}{0.3\textwidth}
            \begin{table}
                \centering
                \textbf{OUTUBRO}\\ \vspace{0.15cm}
                \begin{tblr}{Q[c,m] Q[c,m] Q[c,m] Q[c,m] Q[c,m]}
                    \hline
                    \textbf{S} & \textbf{T} & \textbf{Q} & \textbf{Q} & \textbf{S} \\
                    \hline
                    &  & \aula01 & \aula02 & 03\\
                    06 & 07 & \aula08 & \aula09 & 10\\
                    13 & 14 & \feriado15 & \aula16 & 17\\
                    20 & 21 & \aula22 & \aula\esta{23} & 24\\
                    27 & 28 & \aula29 & \prova30 & 31\\
                    \hline
                \end{tblr}
            \end{table}
        \end{column}
        
        \begin{column}{0.7\textwidth}
            \begin{itemize}
                \justifying
                \item \textbf{Ferramentas de apoio ao desenvolvimento}: IDEs, sistemas de versionamento, bibliotecas auxiliares.
            \end{itemize}
        \end{column}
    \end{columns}
\end{frame}

\begin{frame}{Calendário}
    \centering
    \begin{tblr}{c c c}
        \aula AULA & \feriado FERIADO & \prova AVALIAÇÃO
    \end{tblr}
    
    \begin{columns}
        \begin{column}{0.3\textwidth}
            \begin{table}
                \centering
                \textbf{OUTUBRO}\\ \vspace{0.15cm}
                \begin{tblr}{Q[c,m] Q[c,m] Q[c,m] Q[c,m] Q[c,m]}
                    \hline
                    \textbf{S} & \textbf{T} & \textbf{Q} & \textbf{Q} & \textbf{S} \\
                    \hline
                    &  & \aula01 & \aula02 & 03\\
                    06 & 07 & \aula08 & \aula09 & 10\\
                    13 & 14 & \feriado15 & \aula16 & 17\\
                    20 & 21 & \aula22 & \aula23 & 24\\
                    27 & 28 & \aula\esta{29} & \prova30 & 31\\
                    \hline
                \end{tblr}
            \end{table}
        \end{column}
        
        \begin{column}{0.7\textwidth}
            \begin{itemize}
                \justifying
                \item Prática.
            \end{itemize}
        \end{column}
    \end{columns}
\end{frame}

\begin{frame}{Calendário}
    \centering
    \begin{tblr}{c c c}
        \aula AULA & \feriado FERIADO & \prova AVALIAÇÃO
    \end{tblr}
    
    \begin{columns}
        \begin{column}{0.3\textwidth}
            \begin{table}
                \centering
                \textbf{OUTUBRO}\\ \vspace{0.15cm}
                \begin{tblr}{Q[c,m] Q[c,m] Q[c,m] Q[c,m] Q[c,m]}
                    \hline
                    \textbf{S} & \textbf{T} & \textbf{Q} & \textbf{Q} & \textbf{S} \\
                    \hline
                    &  & \aula01 & \aula02 & 03\\
                    06 & 07 & \aula08 & \aula09 & 10\\
                    13 & 14 & \feriado15 & \aula16 & 17\\
                    20 & 21 & \aula22 & \aula23 & 24\\
                    27 & 28 & \aula29 & \prova\esta{30} & 31\\
                    \hline
                \end{tblr}
            \end{table}
        \end{column}
        
        \begin{column}{0.7\textwidth}
            \Large\centering Segunda Avaliação
        \end{column}
    \end{columns}
\end{frame}

\begin{frame}{Calendário}
    \centering
    \begin{tblr}{c c c}
        \aula AULA & \feriado FERIADO & \prova AVALIAÇÃO
    \end{tblr}
    
    \begin{columns}
        \begin{column}{0.3\textwidth}
            \begin{table}
                \centering
                \textbf{NOVEMBRO}\\ \vspace{0.15cm}
                \begin{tblr}{Q[c,m] Q[c,m] Q[c,m] Q[c,m] Q[c,m]}
                    \hline
                    \textbf{S} & \textbf{T} & \textbf{Q} & \textbf{Q} & \textbf{S} \\
                    \hline
                    03 & 04 & \aula\esta{05} & \aula06 & 07\\
                    10 & 11 & \aula12 & \aula13 & 14\\
                    17 & 18 & \aula19 & \feriado20 & 21\\
                    24 & 25 & \aula26 & \aula27 & 28\\
                    \hline
                \end{tblr}
            \end{table}
        \end{column}
        
        \begin{column}{0.7\textwidth}
            \begin{itemize}
                \justifying
                \item \textbf{Funcionamento de uma aplicação com banco de dados}: modelo cliente-servidor.
            \end{itemize}
        \end{column}
    \end{columns}
\end{frame}

\begin{frame}{Calendário}
    \centering
    \begin{tblr}{c c c}
        \aula AULA & \feriado FERIADO & \prova AVALIAÇÃO
    \end{tblr}
    
    \begin{columns}
        \begin{column}{0.3\textwidth}
            \begin{table}
                \centering
                \textbf{NOVEMBRO}\\ \vspace{0.15cm}
                \begin{tblr}{Q[c,m] Q[c,m] Q[c,m] Q[c,m] Q[c,m]}
                    \hline
                    \textbf{S} & \textbf{T} & \textbf{Q} & \textbf{Q} & \textbf{S} \\
                    \hline
                    03 & 04 & \aula05 & \aula\esta{06} & 07\\
                    10 & 11 & \aula12 & \aula13 & 14\\
                    17 & 18 & \aula19 & \feriado20 & 21\\
                    24 & 25 & \aula26 & \aula27 & 28\\
                    \hline
                \end{tblr}
            \end{table}
        \end{column}
        
        \begin{column}{0.7\textwidth}
            \begin{itemize}
                \justifying
                \item \textbf{Modelagem básica de dados}: entidades, atributos e relacionamentos.
            \end{itemize}
        \end{column}
    \end{columns}
\end{frame}

\begin{frame}{Calendário}
    \centering
    \begin{tblr}{c c c}
        \aula AULA & \feriado FERIADO & \prova AVALIAÇÃO
    \end{tblr}
    
    \begin{columns}
        \begin{column}{0.3\textwidth}
            \begin{table}
                \centering
                \textbf{NOVEMBRO}\\ \vspace{0.15cm}
                \begin{tblr}{Q[c,m] Q[c,m] Q[c,m] Q[c,m] Q[c,m]}
                    \hline
                    \textbf{S} & \textbf{T} & \textbf{Q} & \textbf{Q} & \textbf{S} \\
                    \hline
                    03 & 04 & \aula05 & \aula06 & 07\\
                    10 & 11 & \aula\esta{12} & \aula13 & 14\\
                    17 & 18 & \aula19 & \feriado20 & 21\\
                    24 & 25 & \aula26 & \aula27 & 28\\
                    \hline
                \end{tblr}
            \end{table}
        \end{column}
        
        \begin{column}{0.7\textwidth}
            \begin{itemize}
                \justifying
                \item Prática.
            \end{itemize}
        \end{column}
    \end{columns}
\end{frame}

\begin{frame}{Calendário}
    \centering
    \begin{tblr}{c c c}
        \aula AULA & \feriado FERIADO & \prova AVALIAÇÃO
    \end{tblr}
    
    \begin{columns}
        \begin{column}{0.3\textwidth}
            \begin{table}
                \centering
                \textbf{NOVEMBRO}\\ \vspace{0.15cm}
                \begin{tblr}{Q[c,m] Q[c,m] Q[c,m] Q[c,m] Q[c,m]}
                    \hline
                    \textbf{S} & \textbf{T} & \textbf{Q} & \textbf{Q} & \textbf{S} \\
                    \hline
                    03 & 04 & \aula05 & \aula06 & 07\\
                    10 & 11 & \aula12 & \aula\esta{13} & 14\\
                    17 & 18 & \aula19 & \feriado20 & 21\\
                    24 & 25 & \aula26 & \aula27 & 28\\
                    \hline
                \end{tblr}
            \end{table}
        \end{column}
        
        \begin{column}{0.7\textwidth}
            \begin{itemize}
                \justifying
                \item \textbf{Conexão servidor–banco de dados}: drivers, bibliotecas de acesso.
            \end{itemize}
        \end{column}
    \end{columns}
\end{frame}

\begin{frame}{Calendário}
    \centering
    \begin{tblr}{c c c}
        \aula AULA & \feriado FERIADO & \prova AVALIAÇÃO
    \end{tblr}
    
    \begin{columns}
        \begin{column}{0.3\textwidth}
            \begin{table}
                \centering
                \textbf{NOVEMBRO}\\ \vspace{0.15cm}
                \begin{tblr}{Q[c,m] Q[c,m] Q[c,m] Q[c,m] Q[c,m]}
                    \hline
                    \textbf{S} & \textbf{T} & \textbf{Q} & \textbf{Q} & \textbf{S} \\
                    \hline
                    03 & 04 & \aula05 & \aula06 & 07\\
                    10 & 11 & \aula12 & \aula13 & 14\\
                    17 & 18 & \aula\esta{19} & \feriado20 & 21\\
                    24 & 25 & \aula26 & \aula27 & 28\\
                    \hline
                \end{tblr}
            \end{table}
        \end{column}
        
        \begin{column}{0.7\textwidth}
            \begin{itemize}
                \justifying
                \item \textbf{Consultas SQL em aplicações web}: SELECT, INSERT, UPDATE, DELETE.
            \end{itemize}
        \end{column}
    \end{columns}
\end{frame}

\begin{frame}{Calendário}
    \centering
    \begin{tblr}{c c c}
        \aula AULA & \feriado FERIADO & \prova AVALIAÇÃO
    \end{tblr}
    
    \begin{columns}
        \begin{column}{0.3\textwidth}
            \begin{table}
                \centering
                \textbf{NOVEMBRO}\\ \vspace{0.15cm}
                \begin{tblr}{Q[c,m] Q[c,m] Q[c,m] Q[c,m] Q[c,m]}
                    \hline
                    \textbf{S} & \textbf{T} & \textbf{Q} & \textbf{Q} & \textbf{S} \\
                    \hline
                    03 & 04 & \aula05 & \aula06 & 07\\
                    10 & 11 & \aula12 & \aula13 & 14\\
                    17 & 18 & \aula19 & \feriado20 & 21\\
                    24 & 25 & \aula\esta{26} & \aula27 & 28\\
                    \hline
                \end{tblr}
            \end{table}
        \end{column}
        
        \begin{column}{0.7\textwidth}
            \begin{itemize}
                \justifying
                \item Prática.
            \end{itemize}
        \end{column}
    \end{columns}
\end{frame}

\begin{frame}{Calendário}
    \centering
    \begin{tblr}{c c c}
        \aula AULA & \feriado FERIADO & \prova AVALIAÇÃO
    \end{tblr}
    
    \begin{columns}
        \begin{column}{0.3\textwidth}
            \begin{table}
                \centering
                \textbf{NOVEMBRO}\\ \vspace{0.15cm}
                \begin{tblr}{Q[c,m] Q[c,m] Q[c,m] Q[c,m] Q[c,m]}
                    \hline
                    \textbf{S} & \textbf{T} & \textbf{Q} & \textbf{Q} & \textbf{S} \\
                    \hline
                    03 & 04 & \aula05 & \aula06 & 07\\
                    10 & 11 & \aula12 & \aula13 & 14\\
                    17 & 18 & \aula19 & \feriado20 & 21\\
                    24 & 25 & \aula26 & \aula\esta{27} & 28\\
                    \hline
                \end{tblr}
            \end{table}
        \end{column}
        
        \begin{column}{0.7\textwidth}
            \begin{itemize}
                \justifying
                \item \textbf{Boas práticas na integração banco de dados – servidor web}: segurança e desempenho.
            \end{itemize}
        \end{column}
    \end{columns}
\end{frame}

\begin{frame}{Calendário}
    \centering
    \begin{tblr}{c c c}
        \aula AULA & \feriado FERIADO & \prova AVALIAÇÃO
    \end{tblr}
    
    \begin{columns}
        \begin{column}{0.3\textwidth}
            \begin{table}
                \centering
                \textbf{DEZEMBRO}\\ \vspace{0.15cm}
                \begin{tblr}{Q[c,m] Q[c,m] Q[c,m] Q[c,m] Q[c,m]}
                    \hline
                    \textbf{S} & \textbf{T} & \textbf{Q} & \textbf{Q} & \textbf{S} \\
                    \hline
                    01 & 02 & \aula\esta{03} & \prova04 & 05\\
                    08 & 09 & \prova10 & 11 & \feriado12\\
                    15 & 16 & 17 & 18 & 19\\
                    22 & 23 & 24 & 25 & 26\\
                    29 & 30 & 31 &    &   \\
                    \hline
                \end{tblr}
            \end{table}
        \end{column}
        
        \begin{column}{0.7\textwidth}
            \begin{itemize}
                \item Prática.
            \end{itemize}
        \end{column}
    \end{columns}
\end{frame}

\begin{frame}{Calendário}
    \centering
    \begin{tblr}{c c c}
        \aula AULA & \feriado FERIADO & \prova AVALIAÇÃO
    \end{tblr}
    
    \begin{columns}
        \begin{column}{0.3\textwidth}
            \begin{table}
                \centering
                \textbf{DEZEMBRO}\\ \vspace{0.15cm}
                \begin{tblr}{Q[c,m] Q[c,m] Q[c,m] Q[c,m] Q[c,m]}
                    \hline
                    \textbf{S} & \textbf{T} & \textbf{Q} & \textbf{Q} & \textbf{S} \\
                    \hline
                    01 & 02 & \aula03 & \prova\esta{04} & 05\\
                    08 & 09 & \prova10 & 11 & \feriado12\\
                    15 & 16 & 17 & 18 & 19\\
                    22 & 23 & 24 & 25 & 26\\
                    29 & 30 & 31 &    &   \\
                    \hline
                \end{tblr}
            \end{table}
        \end{column}
        
        \begin{column}{0.7\textwidth}
            \Large\centering Terceira Avaliação
        \end{column}
    \end{columns}
\end{frame}

\begin{frame}{Calendário}
    \centering
    \begin{tblr}{c c c}
        \aula AULA & \feriado FERIADO & \prova AVALIAÇÃO
    \end{tblr}
    
    \begin{columns}
        \begin{column}{0.3\textwidth}
            \begin{table}
                \centering
                \textbf{DEZEMBRO}\\ \vspace{0.15cm}
                \begin{tblr}{Q[c,m] Q[c,m] Q[c,m] Q[c,m] Q[c,m]}
                    \hline
                    \textbf{S} & \textbf{T} & \textbf{Q} & \textbf{Q} & \textbf{S} \\
                    \hline
                    01 & 02 & \aula03 & \prova04 & 05\\
                    08 & 09 & \prova\esta{10} & 11 & \feriado12\\
                    15 & 16 & 17 & 18 & 19\\
                    22 & 23 & 24 & 25 & 26\\
                    29 & 30 & 31 &    &   \\
                    \hline
                \end{tblr}
            \end{table}
        \end{column}
        
        \begin{column}{0.7\textwidth}
            \Large\centering Avaliação Final
        \end{column}
    \end{columns}
\end{frame}

%===============================================================================
% FIM ==========================================================================
%===============================================================================

\begin{frame}
    \centering
    \Large
    FIM
\end{frame}
    
\end{document}
